\section{Estimaci\'on de la se\~nal enviada}

Es necesario hacer una estimaci\'on de dos cosas distintas. Por un lado se debe estimar el canal ($\vec{h}$) y por otro lado, la imagen original  ($\vec{s}$).

%-----------------------------------------------------------
\subsection{Estimaci\'on del canal de comunicaci\'on}

Primero se hace una estimaci\'on de $\vec{h}$ con el m\'etodo de cuadrados m\'inimos, empleando la descomposici\'on de Cholesky, como se explica a continuaci\'on.  Para averiguar $\vec{h}$ a partir de la ecuaci\'on (REFERENCIAAAAAAAAAAAAAAA: r=hs+N), se considera que la media del ruido $N$ es cero, y as\'i se obtiene una nueva ecuaci\'on a partir de la cu\'al se aproceder\'a:

\begin{equation} 
\vec{r} = S \vec{h}  % (2)
\end{equation} 

Para utilizar el m\'etodo de Cholesky en una ecuaci\'on del tipo $\vec{b} = A \vec{h}$ es necesario que la matr\'iz $A$ sea cuadrada, sim\'etrica y definida positiva. Pero en nuestro caso,  en su lugar tenemos a la matr\'iz $S$ que no es  cuadrada. Como necesitamos una matr\'iz que cumpla estas condiciones, multiplicamos ambos lados de la ecuaci\'on (REFERENCIAAAAAA 2) por $S^T$:

\begin{equation} 
S^T \vec{r} = S^T S \vec{h}  % (3)
\end{equation} 

Llamamos  $S^T \vec{r} = \vec{b}$ y  $S^T S = A$, y obtenemos la expresi\'on:

\begin{equation} 
\vec{b} = A \vec{h}  % (4)
\end{equation} 

En esta nueva ecuaci\'on (REFERENCIAAAAAAAAAA 4), la matr\'iz A cumple con las caracter\'isticas para implementar el m\'etodo de Cholesky, el cual consiste en encontrar las matrices $L$ y $L^T$ triangulares, tales que $A = L L^T$; para simplificar la resoluci\'on de la ecuaci\'on. Una vez averiguadas las matrices  $L$ y $L^T$,en la ecuaci\'on (REFERENCIAAAAAAAAAA (4) reemplazamos la matr\'iz $A$ por el producto entre ellas y obtenemos:

\begin{equation} 
\vec{b} = L L^T \vec{h}  % (5)
\end{equation} 

Llamamos $\vec{y} = L^T \vec{h}$ y lo reemplazamos en la ecuaci\'on (REFERENCIAAAAAAA 5), de modo que:

\begin{equation} 
\vec{b} = L \vec{y}  % (6)
\end{equation} 

A partir de la ecuaci\'on (REFERENCIAAAAAAA 6) obtenemos $ \vec{y}$ realizando sustitucci\'on Forward. Finralmente, dado que, como se mencion\'o previamente:

 \begin{equation} 
\vec{y} = L^T \vec{h} %(8)
\end{equation}

se despeja $\vec{h}$ de la ecuaci\'on (rEFERENCIAAAAAAAA 8) haciendo una sustituci\'on Backward. La ventaja de implementar el m\'etodo de Cholesky para la ecuaci\'on resuelta es que se terminan resolviendo dos ecuaciones en las que la matr\'iz involucrada (ya sea $L$ o $L^T$) es una matr\'iz triangulal.

%-----------------------------------------------------------
\subsection{Estimaci\'on de la imagen original}

La estimaci\'on de la imagen original $\vec{s}$, se hace a partir de la ecuaci\'on (REFERENCIAAAAAA R=hS+n), de una manera distinta a la implementada para la estimaci\'on de  $\vec{s}$. A pesar de que tambi\'en se realiza haciendo cuadrados m\'inimos, asumimos que el nivel de ruido es conocido y usamos otro m\'etodo  conocido como Linear Minimum Square Error (LMMSE). Si bien no es lo correcto, asumimos que las m\'inimas unidades transmitidas en la se\~nal enviada son independientes entre s\'i, simplemente para que la resoluci\'on sea m\'as sensilla. 


%EN ALGUNA PARTE HAY QUE MENCIONAR QUE SE HACE UNA SECUENCIA DE ENTRENAMIENTO!!!!!!!!!!!!!!!!!!!!!%
%%%
Para ver como funciona todo lo explicado hasta aqu\'i, se env\'ia la imagen de Lena con un ruido de un desv\'io est\'andar de 1. A continuaci\'on se observan los resultados de la transmisi\'on de la imagen de Lena en escala de grises.
 
 
\begin{figure}
\includegraphics[scale=0.9]{Imagenes/E32S01}
\centering
\caption{Resultados caso $E=32$ $\sigma = 1$ }
\end{figure}
 
\begin{figure}
\includegraphics[scale=0.9]{Imagenes/E32S10}
\centering
\caption{Resultados caso $E=32$ $\sigma = 10$ }
\end{figure}

\begin{figure}
\includegraphics[scale=0.9]{Imagenes/E32S100}
\centering
\caption{Resultados caso $E=32$ $\sigma = 100$ }
\end{figure}

\begin{figure}
\includegraphics[scale=0.9]{Imagenes/E1024S01}
\centering
\caption{Resultados caso $E=1024$ $\sigma = 1$ }
\end{figure}

\begin{figure}
\includegraphics[scale=0.9]{Imagenes/E1024S10}
\centering
\caption{Resultados caso $E=1024$ $\sigma = 10$ }
\end{figure}

\begin{figure}
\includegraphics[scale=0.9]{Imagenes/E1024S100}
\centering
\caption{Resultados caso $E=1024$ $\sigma = 100$ }
\end{figure}

%explicar los resultados haciendo comparaciones entre los resultados y la imagen real, y entre los resultados y el ruido y entre los resultados distintos debiedo a la variable que se fue cambiando.


