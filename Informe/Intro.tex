
%\begin{document}
\section{Introducci\'on}
Toda se\~nal enviada por un sistema de comunicaci\'on sufre perturbaciones durante el proceso de transmisi\'on, y es por eso que se desea reducir el error de la informaci\'on recibida.
Un modelo discreto sensillo de un sistema de comunicaciones es el siguiente. Peri\'odicamente, cada T segundos el transmisor env\'ia un dato $s_k$ considerando como instante inicial a $t = 0$. Luego, la informaci\'on es modificada por el canal a trav\'es de su "respuesta impulsiva". Esto es la respuesta del sistema (en este caso, el canal) frente a una se\~nal de entrada en particular que se conoce como impulso unitario \'o delta de dirac. Matem\'aticamente, esto permite expresar la salida de un sistema en general como la convoluci\'on de su respuesta impulsiva con la se\~nal de entrada.
A su vez, la se\~nal transmitida es afectada por ruido blanco Gaussiano aditivo , donde $N_k \sim cN(0,\sigma)$. Entonces, peri\'odicamente la informaci\'on recibida es \begin{equation*} r_n = \sum_{k=0}^{L-1} h_k s_{n-k} + N_n\end{equation*} Donde $h$  es la respuesta impulsiva previamente mencionada. Matricialmente, esto se puede expresar como: \begin{equation*} \vec{r} = H \vec{s} + \vec{N} \end{equation*} Siendo $ \vec{r}$ un vector $M$-dimensional,  $ \vec{s}$ y $ \vec{N}$  vectores de $M-1$ dimensiones; mientras que $H$ es una matr\'iz de $M$ x $M$; con $M\geq L$ siendo $L$ la longitud de la respuesta impulsiva del sistema. Otra forma de expresar esto mismo es de la siguiente manera:   \begin{equation*} \vec{r} = S \vec{h} + \vec{N} \end{equation*} Siendo $\vec{h}$ un vector de longitud L.
En nuestro caso, $\vec{r}$  es la se\~nal recibida y $\vec{s}$ la se\~nal enviada, que se quiere recuperar a partir de la recibida. Tomando M = L, se busca el $\vec{s}$  que minimice \begin{equation*} ||H\vec{s}-\vec{r}||_2^2 \end{equation*}. Se estima el canal con el m\'etodo de cuadrados m\'inimos para minimizar: \begin{equation*} ||S\vec{h} - \vec{r}||_2^2.

\section{Respuesta al impulso del canal aleatoria}
Con la finalidad de estimar el canal mediante cuadrados m\'inimos, se genera una respuesta aleatoria al impulso de dicho canal, con L = 5. 
%PONER ACA RESULTADOOOOOOOOOOOOOOOOOOOOOOOOOOOOOOOOOOOOOOOOOOOOOOOOO


\section{Recepci\'on recibida de una imagen con ruido}
Se env\'ia la imagen de Lena con un ruido de un desv\'io est\'andar de 1. A continuaci\'on se observan los resultados de la transmisi\'on de la imagen de Lena en escala de grises.
%PONER ACA LAS FOTOOOOOOOOOOOOOOOOOOOOOOOOOOOSSSSSSSSSSSS!!!!!!

\section{Estimaci\'on de $\vec{h}$ empleando una secuencia de entrenamiento}
%COMPLETAR con lo de los ejs del 3 al 6 y comparar


%\end{document}
